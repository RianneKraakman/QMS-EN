% Options for packages loaded elsewhere
\PassOptionsToPackage{unicode}{hyperref}
\PassOptionsToPackage{hyphens}{url}
%
\documentclass[
]{book}
\usepackage{lmodern}
\usepackage{amssymb,amsmath}
\usepackage{ifxetex,ifluatex}
\ifnum 0\ifxetex 1\fi\ifluatex 1\fi=0 % if pdftex
  \usepackage[T1]{fontenc}
  \usepackage[utf8]{inputenc}
  \usepackage{textcomp} % provide euro and other symbols
\else % if luatex or xetex
  \usepackage{unicode-math}
  \defaultfontfeatures{Scale=MatchLowercase}
  \defaultfontfeatures[\rmfamily]{Ligatures=TeX,Scale=1}
\fi
% Use upquote if available, for straight quotes in verbatim environments
\IfFileExists{upquote.sty}{\usepackage{upquote}}{}
\IfFileExists{microtype.sty}{% use microtype if available
  \usepackage[]{microtype}
  \UseMicrotypeSet[protrusion]{basicmath} % disable protrusion for tt fonts
}{}
\makeatletter
\@ifundefined{KOMAClassName}{% if non-KOMA class
  \IfFileExists{parskip.sty}{%
    \usepackage{parskip}
  }{% else
    \setlength{\parindent}{0pt}
    \setlength{\parskip}{6pt plus 2pt minus 1pt}}
}{% if KOMA class
  \KOMAoptions{parskip=half}}
\makeatother
\usepackage{xcolor}
\IfFileExists{xurl.sty}{\usepackage{xurl}}{} % add URL line breaks if available
\IfFileExists{bookmark.sty}{\usepackage{bookmark}}{\usepackage{hyperref}}
\hypersetup{
  pdftitle={Quantitative Methods and Statistics},
  hidelinks,
  pdfcreator={LaTeX via pandoc}}
\urlstyle{same} % disable monospaced font for URLs
\usepackage{longtable,booktabs}
% Correct order of tables after \paragraph or \subparagraph
\usepackage{etoolbox}
\makeatletter
\patchcmd\longtable{\par}{\if@noskipsec\mbox{}\fi\par}{}{}
\makeatother
% Allow footnotes in longtable head/foot
\IfFileExists{footnotehyper.sty}{\usepackage{footnotehyper}}{\usepackage{footnote}}
\makesavenoteenv{longtable}
\usepackage{graphicx}
\makeatletter
\def\maxwidth{\ifdim\Gin@nat@width>\linewidth\linewidth\else\Gin@nat@width\fi}
\def\maxheight{\ifdim\Gin@nat@height>\textheight\textheight\else\Gin@nat@height\fi}
\makeatother
% Scale images if necessary, so that they will not overflow the page
% margins by default, and it is still possible to overwrite the defaults
% using explicit options in \includegraphics[width, height, ...]{}
\setkeys{Gin}{width=\maxwidth,height=\maxheight,keepaspectratio}
% Set default figure placement to htbp
\makeatletter
\def\fps@figure{htbp}
\makeatother
\setlength{\emergencystretch}{3em} % prevent overfull lines
\providecommand{\tightlist}{%
  \setlength{\itemsep}{0pt}\setlength{\parskip}{0pt}}
\setcounter{secnumdepth}{5}
\usepackage{booktabs}
\usepackage{amsthm}
\makeatletter
\def\thm@space@setup{%
  \thm@preskip=8pt plus 2pt minus 4pt
  \thm@postskip=\thm@preskip
}
\makeatother
\usepackage[]{natbib}
\bibliographystyle{apalike}

\title{Quantitative Methods and Statistics}
\author{true}
\date{Version compiled 08 Oct 2020}

\begin{document}
\maketitle

{
\setcounter{tocdepth}{1}
\tableofcontents
}
\hypertarget{preface}{%
\chapter*{Preface}\label{preface}}
\addcontentsline{toc}{chapter}{Preface}

Data are becoming ever more important, in all parts of society, including in the humanities in academia. The availability of large amounts of digital data (such as text, speech, video, behavioural measurements) raises new research questions, which are typically and often investigated using quantitative methods.
Aimed at humanities researchers and students, this book offers an overview of and introduction into the most important quantitative methods and statistical techniques used in the humanities. The book provides a solid methodological foundation for quantitative research, and it introduces the most commonly used statistical techniques to describe data and to test hypotheses. This will also enable the reader to critically evaluate such quantitative research.

This textbook is being used in the course on \emph{Methods and Statistics 1} at Utrecht University (Linguistics program). The book is also highly suitable for self-study at a basic level, for everybody who wishes to learn more about quantitative methods and statistics.

The main text has been kept free of mathematical derivations and formulas, which are typically not very helpful for humanities scholars and students. Our explanation is rather conceptual, and rich in examples. Where necessary we present derivations and formulas in separate sections.

This book also contains instructions on how to ``do'' the statistical analyses and visualisations, both in SPSS (version 22 or later) and in R (version 3.0 or later). These instructions too are in separate sections.

We would like to thank our co-teachers in various courses for the many discussions and examples that have been used in any shape or form in this textbook. We thank our students for their curiosity and for their sharp eyes in spotting errors and inconsistencies in previous versions.

We are also thankful to
Gerrit Bloothooft,
Margot van den Berg,
Willemijn Heeren,
Caspar van Lissa,
Els Rose,
Tobias Quené,
Kirsten Schutter
and Marijn Struiksma,
for their advice, data, comments and suggestions.

We thank Aleksei Nazarov and Joanna Wall for translating this book from Dutch to English.

Utrecht, October 2020

Hugo Quené, \url{https://www.hugoquene.nl}

Huub van den Bergh, \url{https://www.uu.nl/staff/HHvandenBergh}

\begin{center}\rule{0.5\linewidth}{0.5pt}\end{center}

\hypertarget{notation}{%
\section*{Notation}\label{notation}}
\addcontentsline{toc}{section}{Notation}

Following international usage we use the point as decimal symbol; hence we write \(\frac{3}{2}=1.5\). Note that the decimal symbol may vary between computers and between software packages on the same computer. Check which decimal symbol is used by (each software package on) your computer.

\hypertarget{license}{%
\section*{License}\label{license}}
\addcontentsline{toc}{section}{License}

This document is licensed under the \emph{GNU GPL 3} license (for details see
\url{https://www.gnu.org/licenses/gpl-3.0.en.html}).

\hypertarget{citation}{%
\section*{Citation}\label{citation}}
\addcontentsline{toc}{section}{Citation}

Please cite this work as follows (in APA style):

Quené, H. \& Van den Bergh, H. (2020). \emph{Quantitative Methods and Statistics}.
Retrieved 8 Oct 2020 from \url{https://hugoquene.github.io/QMS-EN/} .

\hypertarget{technical-details}{%
\section*{Technical details}\label{technical-details}}
\addcontentsline{toc}{section}{Technical details}

The original Dutch version of this text has been written in LaTeX, and was then converted to Rmarkdown, using \texttt{pandoc} \citep{pandoc} and the \texttt{bookdown} \citep{R-bookdown} in \href{https://www.rstudio.com}{Rstudio}. The Dutch version is available at \url{https://hugoquene.github.io/KMS-NL/} .\\
The English translation is based on the Dutch LaTeX version (for Part I) and Rmarkdown version (for Parts II and III).\\
Other versions of this textbook (EPUB, PDF, HTML), the source code (in Rmarkdown) of the text including examples, accompanying datasets, and figures as separate files, are all available at \url{https://github.com/hugoquene/QMS-EN/} .

\hypertarget{about-the-authors}{%
\section*{About the authors}\label{about-the-authors}}
\addcontentsline{toc}{section}{About the authors}

Both authors work at the Faculty of Humanities at Utrecht University, the Netherlands.
HQ is professor in Quantitative Methods of Empirical Research in the Humanities, and he is also founding director of the Centre for Digital Humanities at Utrecht University. HvdB is professor in Didactics and Testing of Language Proficiency, and he is also section chair in Dutch Language and Literature at the Dutch National Board of Tests and Examinations (CvTE).

  \bibliography{book.bib,packages.bib,hhmhto.bib,pandoc.bib}

\end{document}
